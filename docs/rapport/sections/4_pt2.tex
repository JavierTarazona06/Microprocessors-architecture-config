

\subsection{Q4 : Impact de la taille de L1 (Cortex-A7, entrées \texttt{large})}
L’objectif est d’évaluer la sensibilité des applications à la hiérarchie mémoire en faisant varier la taille de la cache L1
(1, 2, 4, 8, 16\,KB), tout en conservant le même jeu de données (\texttt{large}) et la même configuration globale.
Les métriques principales sont la performance (\texttt{IPC}, \texttt{numCycles}) et les taux de miss des caches (I-L1, D-L1, L2).
On vérifie que \texttt{simInsts} reste constant pour un même binaire : Dijkstra (\texttt{simInsts}=227\,442\,465) et Blowfish (\texttt{simInsts}=12\,761\,206).
~\\
\paragraph{Performance (IPC et cycles).}
~\
\begin{table}[H]
\centering
\small
\caption{Cortex-A7 (large) --- Performance en fonction de la taille L1}
\label{tab:q4_a7_perf}
\begin{tabular}{|c|cc|cc|}
\hline
\multirow{2}{*}{\textbf{L1 (KB)}} &
\multicolumn{2}{c|}{\textbf{Dijkstra}} &
\multicolumn{2}{c|}{\textbf{Blowfish}} \\
\cline{2-5}
& \textbf{IPC} & \textbf{Cycles ($\times 10^6$)} & \textbf{IPC} & \textbf{Cycles ($\times 10^6$)} \\
\hline
1  & 0.233 & 974.9 & 0.248 & 51.4 \\
2  & 0.242 & 941.5 & 0.255 & 50.0 \\
4  & 0.251 & 905.1 & 0.268 & 47.7 \\
8  & 0.271 & 838.3 & 0.295 & 43.3 \\
16 & 0.277 & 820.6 & 0.296 & 43.1 \\
\hline
\end{tabular}
\end{table}

\paragraph{Comportement cache (miss rates).}
~\
\begin{table}[H]
\centering
\small
\caption{A7 (large) --- Miss rates et branchements (mispred\_rate en \%)}
\label{tab:q4_a7_cache}
\resizebox{\linewidth}{!}{%
\begin{tabular}{|c|ccc|c|ccc|c|}
\hline
\multirow{2}{*}{\textbf{L1 (KB)}} &
\multicolumn{4}{c|}{\textbf{Dijkstra}} &
\multicolumn{4}{c|}{\textbf{Blowfish}} \\
\cline{2-9}
& \textbf{I-L1} & \textbf{D-L1} & \textbf{L2} & \textbf{mispred (\%)} &
  \textbf{I-L1} & \textbf{D-L1} & \textbf{L2} & \textbf{mispred (\%)} \\
\hline
1  & 0.0468 & 0.2171 & 0.0002 & 1.36 & 0.3083 & 0.1936 & 0.0015 & 1.40 \\
2  & 0.0341 & 0.1767 & 0.0002 & 1.36 & 0.1263 & 0.1581 & 0.0024 & 1.40 \\
4  & 0.0122 & 0.1361 & 0.0003 & 1.36 & 0.0040 & 0.1063 & 0.0048 & 1.40 \\
8  & 0.0045 & 0.0677 & 0.0006 & 1.35 & 0.0007 & 0.0227 & 0.0246 & 1.40 \\
16 & 0.0010 & 0.0499 & 0.0009 & 1.35 & 0.0006 & 0.0181 & 0.0318 & 1.40 \\
\hline
\end{tabular}%
}
\end{table}

% -------- Figure 1: IPC + Cycles --------
\begin{figure}[H]
\centering
\begin{subfigure}[b]{0.48\linewidth}
  \centering
  \includegraphics[width=\linewidth]{../../tp4/figures/ex4_large/A7_ipc_vs_l1.png}
  \caption{IPC en fonction de la taille de L1.}
  \label{fig:q4_a7_ipc}
\end{subfigure}
\hfill
\begin{subfigure}[b]{0.48\linewidth}
  \centering
  \includegraphics[width=\linewidth]{../../tp4/figures/ex4_large/A7_cycles_vs_l1.png}
  \caption{\texttt{numCycles} vs taille L1 (performance globale).}
  \label{fig:q4_a7_cycles}
\end{subfigure}
\label{fig:q4_a7_perf}
\end{figure}

% -------- Figure 2: L1 I/D miss --------
\begin{figure}[H]
\centering
\begin{subfigure}[b]{0.48\linewidth}
  \centering
  \includegraphics[width=\linewidth]{../../tp4/figures/ex4_large/A7_il1_miss_vs_l1.png}
  \caption{Miss rate I-L1 vs taille L1.}
  \label{fig:q4_a7_il1}
\end{subfigure}
\hfill
\begin{subfigure}[b]{0.48\linewidth}
  \centering
  \includegraphics[width=\linewidth]{../../tp4/figures/ex4_large/A7_dl1_miss_vs_l1.png}
  \caption{Miss rate D-L1 vs taille L1.}
  \label{fig:q4_a7_dl1}
\end{subfigure}
\label{fig:q4_a7_l1_miss}
\end{figure}

% -------- Figure 3: L2 miss + Mispred --------
\begin{figure}[H]
\centering
\begin{subfigure}[b]{0.48\linewidth}
  \centering
  \includegraphics[width=\linewidth]{../../tp4/figures/ex4_large/A7_l2_miss_vs_l1.png}
  \caption{Miss rate L2 (conditionné aux accès L2) vs taille L1.}
  \label{fig:q4_a7_l2}
\end{subfigure}
\hfill
\begin{subfigure}[b]{0.48\linewidth}
  \centering
  \includegraphics[width=\linewidth]{../../tp4/figures/ex4_large/A7_mispred_vs_l1.png}
  \caption{Taux de mauvaise prédiction (\textit{mispred rate}) vs taille L1.}
  \label{fig:q4_a7_mispred}
\end{subfigure}
\label{fig:q4_a7_l2_branch}
\end{figure}

\paragraph{(A7).}
En augmentant L1 de 1\,KB à 16\,KB, on observe un gain net de performance :
\textbf{IPC} passe de 0.233 à 0.277 pour Dijkstra (+18.8\%) et de 0.248 à 0.296 pour Blowfish (+19.4\%),
avec une baisse correspondante de \textbf{cycles} d’environ 16\% dans les deux cas (Table~\ref{tab:q4_a7_perf}).
Ces gains sont principalement expliqués par la chute des miss rates L1 (Table~\ref{tab:q4_a7_cache}) :
Dijkstra réduit fortement ses misses D-L1 (0.217 $\rightarrow$ 0.050, soit -77\%) mais reste plus pénalisé à cause d’accès moins réguliers,
tandis que Blowfish devient très efficace dès 8\,KB (D-L1 $\approx$ 0.023) avec un rendement décroissant ensuite (8$\rightarrow$16\,KB : +0.6\% seulement en IPC).
Le \textbf{mispred rate} reste quasi constant (environ 1.35\% pour Dijkstra et 1.40\% pour Blowfish), indiquant que l’amélioration provient surtout de la hiérarchie mémoire plutôt que du contrôle.
Ainsi, en termes de performance pure, la meilleure configuration parmi celles testées est \textbf{L1=16\,KB} pour les deux applications sur A7,
avec une saturation visible autour de \textbf{8\,KB} (notamment pour Blowfish).

\subsection{Q5 : Impact de la taille de L1 (Cortex-A5, entrées \texttt{large})}

On étudie l'impact de la \textbf{taille de la cache L1} sur les performances du coeur \textbf{A15}
pour deux applications (\texttt{dijkstra\_large} et \texttt{blowfish\_large}).
Comme le nombre d'instructions est (quasi) constant pour une application donnée
(\texttt{simInsts}), une variation de \texttt{numCycles} se traduit directement par une variation d'IPC
(\(\texttt{IPC}=\texttt{simInsts}/\texttt{numCycles}\)).
On interprète les tendances via les \textbf{miss rates} (I-L1, D-L1, L2) et le \textbf{taux de mauvaise prédiction}
(\texttt{mispred\_rate}).

~\
\paragraph{Performance (IPC et cycles).}
~\
\begin{table}[H]
\centering
\small
\caption{Cortex-A15 (large) --- Performance en fonction de la taille L1}
\label{tab:q5_a15_perf}
\begin{tabular}{|c|cc|cc|}
\hline
\multirow{2}{*}{\textbf{L1 (KB)}} &
\multicolumn{2}{c|}{\textbf{Dijkstra}} &
\multicolumn{2}{c|}{\textbf{Blowfish}} \\
\cline{2-5}
& \textbf{IPC} & \textbf{Cycles ($\times 10^6$)} & \textbf{IPC} & \textbf{Cycles ($\times 10^6$)} \\
\hline
2  & 0.685 & 332.0 & 1.094 & 11.7 \\
4  & 0.749 & 303.6 & 1.191 & 10.7 \\
8  & 0.925 & 245.8 & 1.458 & 8.7 \\
16 & 0.996 & 228.4 & 1.489 & 8.6 \\
32 & 1.161 & 195.8 & 1.636 & 7.8 \\
\hline
\end{tabular}
\end{table}

\paragraph{Comportement cache (miss rates).}
~\
\begin{table}[H]
\centering
\small
\caption{A15 (large) --- Miss rates et branchements (mispred\_rate en \%)}
\label{tab:q5_a15_cache_branch}
\resizebox{\linewidth}{!}{%
\begin{tabular}{|c|ccc|c|ccc|c|}
\hline
\multirow{2}{*}{\textbf{L1 (KB)}} &
\multicolumn{4}{c|}{\textbf{Dijkstra}} &
\multicolumn{4}{c|}{\textbf{Blowfish}} \\
\cline{2-9}
& \textbf{I-L1} & \textbf{D-L1} & \textbf{L2} & \textbf{mispred (\%)} &
  \textbf{I-L1} & \textbf{D-L1} & \textbf{L2} & \textbf{mispred (\%)} \\
\hline
2  & 0.0500 & 0.1703 & 0.0001 & 1.56 & 0.1159 & 0.2062 & 0.0016 & 1.35 \\
4  & 0.0265 & 0.1278 & 0.0002 & 1.55 & 0.0058 & 0.1417 & 0.0027 & 1.43 \\
8  & 0.0116 & 0.0661 & 0.0004 & 1.56 & 0.0008 & 0.0402 & 0.0125 & 1.39 \\
16 & 0.0020 & 0.0465 & 0.0005 & 1.55 & 0.0006 & 0.0349 & 0.0163 & 1.39 \\
32 & 0.0016 & 0.0108 & 0.0022 & 1.55 & 0.0005 & 0.0008 & 0.6820 & 1.35 \\
\hline
\end{tabular}%
}
\end{table}

% --- Figure 1: IPC + cycles
\begin{figure}[H]
\centering
\begin{subfigure}{0.49\linewidth}
  \centering
  \includegraphics[width=\linewidth]{../../tp4/figures/ex4_large/A15_ipc_vs_l1.png}
  \caption{IPC en fonction de la taille de L1.}
\end{subfigure}
\hfill
\begin{subfigure}{0.49\linewidth}
  \centering
  \includegraphics[width=\linewidth]{../../tp4/figures/ex4_large/A15_cycles_vs_l1.png}
  \caption{numCycles vs taille L1 (performance globale).}
\end{subfigure}
\label{fig:q5_a15_perf}
\end{figure}

% --- Figure 2: L1 I/D miss
\begin{figure}[H]
\centering
\begin{subfigure}{0.49\linewidth}
  \centering
  \includegraphics[width=\linewidth]{../../tp4/figures/ex4_large/A15_il1_miss_vs_l1.png}
  \caption{Miss rate I-L1 vs taille L1.}
\end{subfigure}
\hfill
\begin{subfigure}{0.49\linewidth}
  \centering
  \includegraphics[width=\linewidth]{../../tp4/figures/ex4_large/A15_dl1_miss_vs_l1.png}
  \caption{Miss rate D-L1 vs taille L1.}
\end{subfigure}
\label{fig:q5_a15_l1_miss}
\end{figure}

% --- Figure 3: L2 miss + mispred
\begin{figure}[H]
\centering
\begin{subfigure}{0.49\linewidth}
  \centering
  \includegraphics[width=\linewidth]{../../tp4/figures/ex4_large/A15_l2_miss_vs_l1.png}
  \caption{Miss rate L2 (conditionné aux accès L2) vs taille L1.}
\end{subfigure}
\hfill
\begin{subfigure}{0.49\linewidth}
  \centering
  \includegraphics[width=\linewidth]{../../tp4/figures/ex4_large/A15_mispred_vs_l1.png}
  \caption{Taux de mauvaise prédiction (mispred rate) vs taille L1.}
\end{subfigure}
\label{fig:q5_a15_l2_branch}
\end{figure}

\paragraph{(A15).}
Sur A15, l'augmentation de la taille L1 réduit fortement les \textbf{misses L1} (I et D), ce qui diminue
\texttt{numCycles} et augmente l'IPC. Le gain est particulièrement marqué entre 4KB et 8KB (diminution nette des
misses), puis les rendements deviennent décroissants. Le \textbf{taux de mauvaise prédiction} varie très peu avec L1,
ce qui indique que l'amélioration provient majoritairement de la \textbf{hiérarchie mémoire} (et non du front-end branchement).
Le point à retenir est donc que, pour ces workloads \texttt{large}, la performance A15 est fortement corrélée
à la capacité de L1 à absorber le working set (surtout en données pour Dijkstra).